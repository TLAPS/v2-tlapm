\documentclass[a4paper]{article}

\usepackage{fullpage}
\usepackage{amsmath}
\usepackage{amssymb}
\usepackage{listings}

\title{An instantiation algorithm for TLA+ expressions}

\newcommand{\assignment}[1]{\{#1\}}
\newcommand{\inst}[2]{#1 {\leftarrow} #2}
\newcommand{\einst}[3]{#1 \stackrel{#2}{\Leftarrow} #3}
\newcommand{\tlaplus}[0]{{TLA+}}

\begin{document}

\maketitle

\section{Overview}
\label{sec:overview}
\tlaplus{} has two kinds of substitution: instantiation of modules, which
 preserves validity, and beta-reduction of lambda-expressions, which does not
 necessarily preserve validity. Moreover, the two substitutions do not commute:


\section{Old Attempt}

We define two substitutions, one for within the context of enabled and one for
 within.

Syntax conventions:

\begin{tabular}[h]{ll}
  $c,d,\ldots$ & rigid variable \\
  $x,y,\ldots$ & flexible variable \\
  $\overline{x}$ & vector $x_1, \ldots, x_k$ of variables where $k$ is irrelevant \\
  $e$  & expression \\
  $\sigma$  & a map (flexible and rigid) variable to expression \\
  $\xi$  & a renaming i.e. a map variable to variable \\
\end{tabular}

Substitution mappings differ from the identity function only on a finite number
 of values.

\[
\begin{array}[h]{l@{\,\,:=\,\,}lp{7cm}}
  \inst{c}{\sigma} & \sigma{c} & \\
  \inst{x}{\sigma} & \sigma(x) & \\
  \inst{Op}{\sigma} & TODO & \\
  \inst{Op(e_1, \ldots, e_n)}{\sigma}
                   & \inst{Op}{}(\inst{e_1}{\sigma}, \ldots, \inst{e_n}{\sigma})
                       & \mbox{ if Op has the Leibniz property} \\
  \inst{ENABLED(e)}{\sigma} & \exists \overline{y}( \einst{e}{\xi}{\sigma} )
                       & \mbox{where $\xi(y) \not \in FV(e)$ for all $y \in FV(e)$} \\
 & & \mbox{
                            i.e. $\xi$ assigns fresh variables to every variable
                            in $e$ }  \\

  \einst{c}{\xi}{\sigma} & \sigma{c} & \\
  \einst{x}{\xi}{\sigma} & \sigma(x) & \\
  \einst{e'}{\xi}{\sigma} & (\einst{\inst{e}{\xi}}{\xi}{\sigma})' & \\
  \einst{Op}{\xi}{\sigma} & TODO & \\
  \einst{Op(e_1, \ldots, e_n)}{\xi}{\sigma}
                   & \inst{Op}{}(\einst{e_1}{\xi}{\sigma}, \ldots, \einst{e_n}{\xi}{\sigma})
                       & \mbox{ if Op has the Leibniz property} \\
\end{array}
\]

\end{document}
